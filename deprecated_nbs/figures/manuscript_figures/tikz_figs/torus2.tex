\documentclass[tikz,border=7pt]{standalone}
\usetikzlibrary{calc,3d}
\tikzset{
  % ------ define the torus pic
  pics/torus/.style n args={3}{
    code = {
      \begin{scope}
        \edef\shorten{\the\dimexpr-.5\pgflinewidth\relax}
        \providecolor{pgffillcolor}{rgb}{1,1,1}
        \tikzstyle{torus} = [
          line join=round,draw,line width/.expanded={\the\dimexpr2\pgflinewidth+#2*2\relax},
          postaction={draw=pgffillcolor,line width={#2*2},shorten <=\shorten,shorten >=\shorten},
          yscale=cos(#3)
        ]
        \draw[torus] circle(#1);
        \draw[torus] (180:#1) arc (180:360:#1);
      \end{scope}
    }
  },
  % ------ define the circle pic
  pics/circle on torus/.style n args={4}{
    code = {
      \begin{scope}[
        plane x={({cos(#4)},{-cos(#3)*sin(#4)})},
        plane y={(0,{sin(#3)})},
        canvas is plane]
        \draw[xshift=#1, rotate=sign(-#4)*#3] (0:#2) arc(0:180:#2);
        \draw[xshift=#1, rotate=sign(-#4)*#3-180, densely dotted,draw opacity=.5] (0:#2) arc(0:180:#2);
      \end{scope}
    }
  }
}
\begin{document}
  \def\R{1cm} % <-- the big radius
  \def\r{2.8mm} % <-- the small radius
  \def\a{70} % <-- the view perspective angle
  \begin{tikzpicture}[rotate=30,transform shape]
    % draw the torus
    \pic[fill=yellow!20]{torus={\R}{\r}{\a}};
    % draw the circles
    % \pic[red]{circle on torus={\R}{\r}{\a}{45}};
    % \pic[blue]{circle on torus={\R}{\r}{\a}{110}};
    % \pic[violet]{circle on torus={\R}{\r}{\a}{-70}};
  \end{tikzpicture}
\end{document}