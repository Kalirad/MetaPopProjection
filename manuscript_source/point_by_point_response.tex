\documentclass[letterpaper,11pt]{article}
\usepackage{palatino}
\usepackage{mathpazo}
\usepackage{enumitem}
\usepackage{parskip}
\usepackage{geometry} % Required for adjusting page dimensions

\geometry{
	paper=letterpaper, % Change to letterpaper for US letter
	top=1.2cm, % Top margin
	bottom=2cm, % Bottom margin
	left=2cm, % Left margin
	right=2cm, % Right margin
	%showframe, % Uncomment to show how the type block is set on the page
}


\begin{document}

\begin{flushright}
\today
\end{flushright}

\bigskip

\noindent
Dear Dr Rafael D'Andrea and James O'Dwyer,

We are grateful for .... 

\bigskip

\noindent
Yours sincerely,

\bigskip

\noindent
Ralf J. Sommer, PhD and Ata Kalirad, PhD

\newpage

\begin{center}
\LARGE \bf
Response to the Editors' and Reviewers' Comments
\end{center}

\smallskip

\begin{center}
  \emph{Below, we reproduce the editors' and reviewers' comments in their entirety \\ and 
  address them one-by-one.  We numbered them continuously in order \\ of appearance.
  We subdivided and renumbered some individual \\ comments for convenience.}
  \end{center}

\section*{Editors (Dr Rafael D'Andrea and James O'Dwyer)}

\begin{enumerate}[label=E.\arabic*, ref=(E.\arabic*)]

\item I agree with Reviewer \#2 that this paper has potential to contribute to addressing an important unresolved question, but I agree with Reviewer \#1 that in its current form, it is not clear exactly what that contribution is. While I do not think that this paper necessarily needs fresh data or a broader focus, I agree with Reviewer \#1 that all sections need refinement to clarify what questions are being asked and what general insights are being gained. 

\begin{quote}
\textbf{Response:} ...
\end{quote}

\item The main research question, which seems to be whether and how environmental structure modifies the cost of plasticity and competitive outcomes, must be given more context and motivation in the Introduction. While competition is mentioned several times in the Introduction, it is not clear whether this study aims to measure it or its effects until the Results section. The need for a metacommunity approach is not justified until late in the paper. The topic of coexistence between the two strains is not mentioned until well into the Results section on line 256. Indeed, most of the content of the first paragraph and some of the content in the other paragraphs in the Results section belongs in the Introduction.

\begin{quote}
\textbf{Response:} ...
\end{quote}

\item As noted by Reviewer \#1, the difference between cost of plasticity and cost of phenotype needs more careful explanation in the Introduction. I.e., it must be clearly stated that the cost of phenotype is defined as the difference in fitness between NP and P strains under the non-inducing diet whereas the cost of plasticity is the difference in fitness under the inducing diet; then, this choice of definition needs to be explained. Why, for example, shouldn’t these deltas be weighed by the relative frequency with which the species faces each of those environments? The results of this study clearly show that those weights can make all the difference (and indeed this is explicitly noted later on lines 237-238).

\begin{quote}
\textbf{Response:} ...
\end{quote}

\item The Methods section needs more detail, as both reviewers point out. In addition to the reviewers’ comments, I suggest better explaining the difference in life history traits between the plastic and non-plastic strains. For example, it is not clear how the different parameters in matrix U (which is not labeled) relate to the experimental diets. Also, some of the notation is inconsistent. In Table 1, the letter m represents fecundity, while in Eqns. (3) and (6) it is an index for life stage. The letter f is also used to represent fecundity in matrix F, and later frequency (line 173).

\begin{quote}
  \textbf{Response:} ...
  \end{quote}

\item As Reviewer \#1 points out, the observation that temporal and spatial structure in the environment affects competitive outcomes is not very surprising. In light of the existing literature on conditions favoring plasticity (referenced in the third paragraph of the Discussion), this study’s impact would be strengthened by a more systematic exploration of the conditions that give plasticity an advantage in this species, leading to more specific predictions and clear insights. Something along the lines of ``the benefits of plasticity in this species outweigh the costs when the environment has general characteristics XYZ, and this happens because of ABC''.

\begin{quote}
  \textbf{Response:} ...
  \end{quote}

\item In addition to referencing modern coexistence theory, the authors may want to cite the literature on emergent neutrality when discussing the finding of long periods of transient coexistence between the two strains (e.g. Scheffer and van Nes 2006 and more recent papers that cite it).

\begin{quote}
  \textbf{Response:} ...
  \end{quote}

\end{enumerate}

\section*{Reviewer \#1} 

This manuscript uses life-history data from two nematode species to model how their niche differences might play out as competitive differences in a heterogeneous environment. The goal of this modeling is to understand how costs of plasticity might be offset by its benefits in scenarios that mimic the spatial structure and variability of nature.

The Introduction aims for precision about the language around plasticity and costs, yet this section does not clearly define how the authors believe that costs of plasticity should ideally be measured. A “cost of phenotype” is also mentioned but not defined or connected with costs of plasticity. Based on the definitions in the introduction, the designation of cost of plasticity and cost of phenotype in the system used here makes it unclear if the cost of plasticity is an overall cost of maintaining plastic mechanisms or simply a cost from producing a more involved phenotype and directing readers to a results figure doesn’t seem sufficient to clarify this point. Moreover, the intro doesn’t make a clear case for this paper as a standalone contribution. There’s not a very clear statement of a research question, and the motivation seems to largely be tying off loose ends from a previous paper. The clearest statement of results is “We illustrate how environmental heterogeneity in space and time can alleviate the inferred cost of plasticity,” but from this statement I’m confused at whether modeling will reveal reduced costs, or increased benefits, of plasticity. Overall, I think this Intro needs to be refined.



\end{document}